%-------------------------
% Rezume, a latex resume template for developers
% Author : Nanu Panchamurthy
% Based off of: https://github.com/sb2nov/resume
% License : MIT

% Hope this resume template helps you land an awesome job.
%-------------------------


%------------PACKAGES----------------
\documentclass[a4paper,11pt]{article}

\usepackage{verbatim} % reimplements the "verbatim" and "verbatim*" environments

\usepackage{titlesec} % provides an interface to sectioning commands i.e. custom elements

\usepackage{color} % provides both foreground and background color management

\usepackage{enumitem} % provides control over enumerate, itemize and description

\usepackage{fancyhdr} % provides extensive facilities for constructing headers, footers and also controlling their use

\usepackage{tabularx} % defines an environment tabularx, extension of "tabular" with an extra designator x, paragraph like column whose width automatically expands to fill the width of the environment

\usepackage{latexsym} % provides mathematical symbols

\usepackage{marvosym} % provides martin vogel's symbol font which contains various symbols

\usepackage[empty]{fullpage} % sets margins to one inch and removes headers, footers etc..

\usepackage[hidelinks]{hyperref} % removes color and shadow of hyperlinks

\usepackage[normalem]{ulem} % provides "\ul" (uline) command which will break at line breaks

\usepackage[english]{babel} % provides culturally determined typographical rules for wide range of languages
%-----------------------------------------

\input glyphtounicode % converts glyph names to unicode
\pdfgentounicode=1 % ensures pdfs generated are ats readable

%----------FONT OPTIONS-------------------
\usepackage[default]{sourcesanspro} % uses the font source sans pro
\urlstyle{same} % changes url font from default urlfont to font being used by the document
%-----------------------------------------


%----------MARGIN OPTIONS-----------------
\pagestyle{fancy} % set page style to one configured by fancyhdr
\fancyhf{} % clear all header and footer fields

\renewcommand{\headrulewidth}{0in} % sets thickness of linerule under header to zero
\renewcommand{\footrulewidth}{0in} % sets thickness of linerule over footer to zero

\setlength{\tabcolsep}{0in} % sets thickness of column separator in tables to zero
\setlength{\footskip}{4.08003pt} % sets distance between bottom of footer and bottom of last line of text to 4.08003pt

% origin of the document is one inch from the top and from and the left
% oddsidemargin and evensidemargin both refer to the left margin
% right margin is indirectly set using oddsidemargin
\addtolength{\oddsidemargin}{-0.5in}
\addtolength{\topmargin}{-0.5in}

\addtolength{\textwidth}{1.0in} % sets width of text area in the page to one inch
\addtolength{\textheight}{1.0in} % sets height of text area in the page to one inch

\raggedbottom{} % makes all pages the height of current page, no extra vertical space added
\raggedright{} % makes all pages the width of current page, no extra horizontal space added
%------------------------------------------


%--------SECTIONING COMMANDS---------
% \titleformat{<command>}
%   [<shape>]{<format>}{<label>}{<sep>}
%     {<before-code>}[<after-code>]

% command is the sectioning command to be redefined
% shape is the style of the font; scshape stands for small caps style
% format is the format to be applied to whole title- label and text; absent here
% label defines the label
% sep is the horizontal separation between label and title body
% before-code is the code to be executed before
% after-code is the code to be executed after

\titleformat{\section}
  {\scshape\large}{}
    {0em}{\color{blue}}[\color{black}\titlerule\vspace{0pt}]
%-------------------------------------


%--------REDEFINITIONS----------------
% redefines the style of the bullet point
\renewcommand\labelitemii{$\vcenter{\hbox{\tiny$\bullet$}}$}

% redefines the underline depth to 2pt
\renewcommand{\ULdepth}{2pt}
%-------------------------------------


%--------CUSTOM COMMANDS--------------
%\vspace{} defines a vertical space of given size, modifying this in custom commands can help stretch or shrink resume to remove or add content

% resumeItem renders a bullet point
\newcommand{\resumeItem}[1]{
  \item\small{#1}
}

% commands to start and end itemization of resumeItem, rightmargin set to 0.11in to avoid the overflow of resumetItem beyond whatever resumeItemHeading is being used
\newcommand{\resumeItemListStart}{\begin{itemize}[rightmargin=0.11in]}
\newcommand{\resumeItemListEnd}{\end{itemize}}

% resumeSectionType renders a bolded type to be used under a section, used as skill type here, middle element is used to keep ":"s in the same vertical line
\newcommand{\resumeSectionType}[3]{
  \item\begin{tabular*}{0.96\textwidth}[t]{
    p{0.15\linewidth}p{0.02\linewidth}p{0.81\linewidth}
  }
    \textbf{#1} & #2 & #3
  \end{tabular*}\vspace{-2pt}
}

% resumeTrioHeading renders three elements in three columns with second element being italicized and first element bolded, can be used for projects with three elements
\newcommand{\resumeTrioHeading}[3]{
  \item\small{
    \begin{tabular*}{0.96\textwidth}[t]{
      l@{\extracolsep{\fill}}c@{\extracolsep{\fill}}r
    }
      \textbf{#1} & \textit{#2} & #3
    \end{tabular*}
  }
}

% resumeQuadHeading renders four elements in a two columns with the second row being italicized and first element of first row bolded, can be used for experience and projects with four elements
\newcommand{\resumeQuadHeading}[4]{
  \item
  \begin{tabular*}{0.96\textwidth}[t]{l@{\extracolsep{\fill}}r}
    \textbf{#1} & #2 \\
    \textit{\small#3} & \textit{\small #4} \\
  \end{tabular*}
}

% resumeQuadHeadingChild renders the second row of resumeQuadHeading, can be used for experience if different roles in the same company need to added
\newcommand{\resumeQuadHeadingChild}[2]{
  \item
  \begin{tabular*}{0.96\textwidth}[t]{l@{\extracolsep{\fill}}r}
    \textbf{\small#1} & {\small#2} \\
  \end{tabular*}
}

% commands to start and end itemization of resumeQuadHeading, lefmargin for left indent of 0.15in for resumeItems
\newcommand{\resumeHeadingListStart}{
  \begin{itemize}[leftmargin=0.15in, label={}]
}
\newcommand{\resumeHeadingListEnd}{\end{itemize}}
%-------------------------------------------


%__________________RESUME____________________
% You can rearrange sections in any order you may prefer
\begin{document}

%-----------CONTACT DETAILS------------------
% Make sure all the details are correct, you can add more links in the first row of second column if needed

\begin{tabular*}{\textwidth}{l@{\extracolsep{\fill}}r}
  \textbf{\Huge Garfield Lee \vspace{2pt}} & % row = 1, col = 1
  Location: Shanghai, China \\ % row = 1, col = 2
  \href{https://550.moe}{\uline{550.moe}} $|$ % row = 2, col = 1
  \href{https://www.linkedin.com/in/garfieldlee}{\uline{LinkedIn}} $|$ % row = 2, col = 1
  \href{https://github.com/Garfield550}{\uline{GitHub}} & % row = 2, col = 1
  Email: \href{mailto:career@550.moe}{\uline{career@550.moe}} \\ % row = 2, col = 2
\end{tabular*}
%--------------------------------------------


%-----------SUMMARY--------------------------
% Keep this short, simple and straigth to point

\section{Web Developer}
\small{
  \textbf{5 years of experience} as a web developer, I am familiar with \textbf{React, Next.js, and TypeScript}. I also have experience in Web3 and blockchain development, having worked on projects such as an \textbf{NFT marketplace}, an \textbf{NFT minting platform}, and a \textbf{blog platform based on Web3 technology}. Enjoy trying new web technologies and frameworks, actively promoting and using them in team projects.
}
%--------------------------------------------


%--------------SKILLS------------------------
% Add or remove resumeSectionTypes according to your needs

\section{Technical Skills}
  \resumeHeadingListStart{}
    \resumeSectionType{Languages}{:}{TypeScript, HTML, CSS, C Sharp, Zig}
    \resumeSectionType{Frameworks}{:}{React, Next.js, NestJS, Node.js}
    \resumeSectionType{Libraries}{:}{Zod, Wagmi, Tailwind CSS, Prisma}
    \resumeSectionType{Databases}{:}{PostgreSQL, MySQL}
    \resumeSectionType{Dev Tools}{:}{ESLint, Visual Studio Code, Git, GitHub Copilot}
  \resumeHeadingListEnd{}
%--------------------------------------------


%-----------PROJECTS--------------------------
% Use resumeQuadHeading if four elements are feasible (ex: demo video link), else use resumeTrioHeading. Keep the bullet points simple and concise and try to cover wide variety of skills you have used to build these projects

\section{Projects}
\resumeHeadingListStart{}
  \resumeTrioHeading{\href{https://project1.com}{\uline{Project 1}}}{React.js, Redux, PHP, MySQL Git}{\href{https://proect1.com/source-code/}{\uline{Source Code}}}
    \resumeItemListStart{}
      \resumeItem{Designed and developed a clean and modern website using \textbf{HTML, CSS, and JavaScript}}
      \resumeItem{Optimized website for \textbf{speed and user experience}}
      \resumeItem{Utilized \textbf{responsive design} to ensure compatibility across all devices}
      \resumeItem{Deployed on GitHub pages via GitHub Actions}
    \resumeItemListEnd{}

    \resumeTrioHeading{Project 2}{Node.js, Express, JavaScript, Git}{\href{https:project2.com/source-code}{\uline{Source Code}}}
    \resumeItemListStart{}
      \resumeItem{A \textbf{CRUD application} exposed using a RESTful API made with Node.js}
      \resumeItem{Exposed POST, GET, PATCH and DELETE HTTP methods using \textbf{Express}}
    \resumeItemListEnd{}
\resumeHeadingListEnd{}
%--------------------------------------------


%-----------EXPERIENCE-----------------------
% Distill all your talking points to small bullet points which follow the pattern "challenge-action-result" for maximum efficiency. Try to quantify (use numbers) your points whenver possible, highlist words of importance

\section{Experience}
\resumeHeadingListStart{}

  \resumeQuadHeading{Web Developer}{Nov 2020 -- Jun 2023}
  {Qiubi Technology (Shanghai) Co., Ltd.}{Remote -- Shanghai, China}

    \resumeItemListStart{}
      \resumeItem{Adding new features for an enhanced Civitai project, to allow users to use Ethereum wallet login, I created a custom \textbf{NextAuth.js CredentialsProvider} and used \textbf{SIWE} for signature verification. \textbf{Web3Modal and Wagmi} were used on the frontend to connect to the wallet. To implement the hosting of Gradio projects, I developed a backend service using \textbf{H3.js}, validated environment variables and API parameters with \textbf{Zod}, and wrapped Git and Docker commands with \textbf{execa} to package Gradio projects from GitHub or Hugging Face into Docker images, hosted on the server with custom ports and domains so that the frontend can display Gradio projects.}

      \resumeItem{Collaborated with our CTO to build a backend service for a web3 blog toolkit. Using \textbf{NestJS, TypeORM, MySQL, and TypeScript}, I completed the backend API for blog information, storage targets, and theme configuration, and integrated \textbf{Arweave} to store user post's metadata. I built a website generator based on \textbf{Hexo.js} using \textbf{Node.js}, which can automatically generate and deploy Hexo websites according to the presets provided by the backend.}

      \resumeItem{I assisted a web3 MOBA game company, and have developed NFT minting websites and user center websites. I used \textbf{Vue.js, Headless UI, UnoCSS}(a Tailwind CSS alternative)\textbf{, and TypeScript} to build the minting website. As Vue does not have a good wallet connection framework, I utilized the \textbf{Vue Composition} feature to encapsulate \textbf{Web3Modal HTML and Wagmi Core} to achieve similar functionality as Web3Modal React. During my maintenance of this project, the company successfully conducted four rounds of NFT issuance activities. Together with my colleagues, we built the user center website using \textbf{React and TypeScript}. I split UI elements into components for efficient reuse and also helped optimize some \textbf{React Hooks} to improve page performance. To better reuse UI components and share business logic, I used \textbf{monorepo} reorganized code from multiple projects into independent packages which solved problems such as copy-pasting and logic synchronization.}
    \resumeItemListEnd{}

  \resumeQuadHeading{Web Developer}{Nov 2018 -- Oct 2020}
  {Shanghai Hayi Information Technology Co., Ltd.}{Shanghai, China}

    \resumeItemListStart{}
      \resumeItem{I am responsible for the development of the company's WeChat Mini Program project. I utilize \textbf{Taro} to standardize the development process and improve code reuse rate. With Taro's comprehensive support for React, I can easily handle the development and maintenance of different mini-program platforms, effectively reducing development costs. In addition, I also promote \textbf{TypeScript and ESLint} to improve code quality and reduce bugs during production. With my assistance, the company's projects have achieved more efficient development and higher quality code implementation.}

      \resumeItem{I am responsible for customizing the October CMS theme for the company, using \textbf{TypeScript and Sass} to write frontend logic and styles, and using \textbf{Webpack and PostCSS} to package scripts and resource files. By removing messy script tags and style tags, I have reduced issues such as functional failures and style conflicts while increasing code maintainability. Additionally, by integrating commonly used October CMS components and styles into an npm package, I have achieved better code sharing which has improved team development efficiency. With my assistance, our clients have received a better user experience with higher quality project implementation.}
    \resumeItemListEnd{}

  \resumeQuadHeadingChild{Software Developer Intern}{Aug 2017 -- Feb 2018}

    \resumeItemListStart{}
      \resumeItem{Assisted lead developer maintaining and developing new features using C Sharp and ASP.NET Core.}
    \resumeItemListEnd{}

\resumeHeadingListEnd{}
%---------------------------------------------


%-----------EDUCATION-------------------------
% Mention your CGPA, if its good, in the first row of second column

\section{Education}
  \resumeHeadingListStart{}
    \resumeQuadHeading{Shandong Vocational College of Information Technology}{Weifang, Shandong, China}
    {Software Engineering}{Jul 2015 -- Jul 2018}
  \resumeHeadingListEnd{}
%---------------------------------------------


%----------------OTHERS----------------------
% You can add your acheivements, accolades, certifications etc. here.

%\section{Certifications}
%  \resumeItemListStart{}
%    \resumeItem{\href{https://dummy-certification.com}{\uline{Certified Web Developer by the W3C}}}
%    \resumeItem{\href{https://dummy-certification.com}{\uline{Microsoft Certified: Azure Developer Associate}}}
%    \resumeItem{\href{https://dummy-certification.com}{\uline{AWS Certified Developer - Associate}}}
%  \resumeItemListEnd{}
%--------------------------------------------

\end{document}
